% @file paper.tex
% @brief 2022高等数学I(2)期末考题A
% @author 24bit-xjkp
% @email 2283572185@qq.com

\section{填空题:本题共 8 小题,每小题 3 分,共 24 分}

\begin{question}[points = 3]
    $\lim\limits_{(x, y) \to (0, 0)} \frac{2 - \sqrt{xy + 4}}{xy} = $ \fillin[$-\frac{1}{4}$];
\end{question}
\begin{solution}
    令 $t = xy \quad \therefore \lim\limits_{(x, y) \to (0, 0)} \frac{2 - \sqrt{xy + 4}}{xy} = \lim\limits_{t \to 0} \frac{2 - \sqrt{t + 4}}{t}$
    $= \lim\limits_{t \to 0} -\frac{1}{2\sqrt{t + 4}} = -\frac{1}{4}$
\end{solution}

\begin{question}[points = 3]
    若函数 $f(x, y) = x^2y + e^{xy},\, M(1,1)$ ,则 $\mathrm{d}f|_M = $ \fillin[$(2 + e)\mathrm{d}x + (1 + e)\mathrm{d}y$];
\end{question}
\begin{solution}
    $\because \frac{\partial f}{\partial x} = 2xy + ye^{xy} \quad \frac{\partial f}{\partial y} = x^2 + xe^{xy}$

    $\therefore \frac{\partial f}{\partial x}|_M = 2 + e \quad \frac{\partial f}{\partial y}|_M = 1 + e \qquad$
    $\therefore \mathrm{d}f|_M = (2 + e)\mathrm{d}x + (1 + e)\mathrm{d}y$
\end{solution}

\begin{question}[points = 3]
    若函数 $f(x, y, z) = xyz$ ,则 $\mathrm{div}(\mathrm{grad}f) = $ \fillin[$0$];
\end{question}
\begin{solution}
    $\because \frac{\partial^2 f}{\partial x^2} = \frac{\partial^2 f}{\partial y^2} = \frac{\partial^2 f}{\partial z^2} = 0 \qquad$
    $\therefore \triangle f = 0$
\end{solution}

\begin{question}[points = 3]
    设 $\Sigma$ 是上球面 $z = \sqrt{4 - x^2 - y^2}$ ,则
    $\iint\limits_\Sigma \left(\frac{1}{\sqrt{x^2 + y^2 + z^2}} + xyz\right)\mathrm{d}S = $ \fillin[$4\pi$];
\end{question}
\begin{solution}
    $\because \Sigma: x^2 + y^2 + z^2 = 4,\, z \geq 0$ ,关于平面 $xOz$ 对称

    $\therefore \iint\limits_\Sigma xyz\mathrm{d}S = 0 \quad$
    $\iint\limits_\Sigma \left(\frac{1}{\sqrt{x^2 + y^2 + z^2}}\right)\mathrm{d}S = \frac{1}{2} \iint\limits_\Sigma\mathrm{d}S$
    $= \frac{1}{2} \times \frac{1}{2} \times 4\pi \times 2^2 = 4\pi$

    $\therefore \iint\limits_\Sigma \left(\frac{1}{\sqrt{x^2 + y^2 + z^2}} + xyz\right)\mathrm{d}S = 4\pi + 0 = 4\pi$
\end{solution}

\begin{question}[points = 3]
    设 $L$ 为 $x^2 + y^2 = 1$ 正向封闭曲线,则曲线积分 $\oint_L \frac{y\mathrm{d}x + x\mathrm{d}y}{x^2 + y^2} = $ \fillin[$0$];
\end{question}
\begin{solution}
    $\because \oint_L \frac{y\mathrm{d}x + x\mathrm{d}y}{x^2 + y^2} = \oint_L y\mathrm{d}x + x\mathrm{d}y$

    $\therefore$ 令 $P = y \quad Q = x \qquad \therefore \frac{\partial P}{\partial y} = \frac{\partial Q}{\partial x} \qquad$
    $\therefore \oint_L y\mathrm{d}x + x\mathrm{d}y = 0$;
\end{solution}

\begin{question}[points = 3]
    如果幂级数 $\sum\limits_{n = 0}^{\infty} a_nx^n$ 当 $x = 2$ 时收敛,
    则级数 $\sum\limits_{n = 0}^{\infty} (-1)^na_n$ \fillin[绝对收敛] (填条件收敛、绝对收敛、发散);
\end{question}
\begin{solution}
    $\because \sum\limits_{n = 0}^{\infty} a_nx^n$ 当 $x = 2$ 时收敛

    $\therefore R \geq 2 \qquad \therefore \rho = \lim\limits_{n \to \infty} \left|\frac{a_{n + 1}}{a_n}\right| \leq \frac{1}{2} \qquad$
    $\therefore \sum\limits_{n = 0}^{\infty} (-1)^na_n$ 绝对收敛;
\end{solution}

\begin{question}[points = 3]
    微分方程 $y^{\prime} + y\tan{x} = \cos{x}$ 的通解为 \fillin[$y = (x + C)\cos{x}$];
\end{question}
\begin{solution}
    $\because y^{\prime} + y\tan{x} = \cos{x}$

    $\therefore \int \frac{\mathrm{d}y}{y} = \int -\tan{x} + \frac{\cos{x}}{y}\mathrm{d}x \qquad$
    $\therefore \ln{y} = \ln{\cos{x}} + C(x) \qquad \therefore y = C(x)\cos{x}$

    $\therefore C^{\prime}\cos{x} - C\sin{x} + C\sin{x} = \cos{x} \qquad \therefore C^{\prime} = 1 \qquad \therefore C(x) = x + C$

    $\therefore y = (x + C)\cos{x}$
\end{solution}

\begin{question}[points = 3]
    微分方程 $(1 - x^2)y^{\prime\prime} - xy^{\prime} = 0$ 满足初始条件
    $y|_{x = 0} = 0,\, y^{\prime}|_{x = 0} = 1$ 的特解为 \fillin[$y = \arcsin{x}$];
\end{question}
\begin{solution}
    令 $u = y^{\prime} \qquad \therefore u|_{x = 0} = 1$

    $\because (1 - x^2)u^{\prime} - xu = 0 \qquad \therefore \int \frac{\mathrm{d}u}{u} = \int \frac{x}{1 - x^2}\mathrm{d}x \qquad$
    $\therefore \ln{u} = -\frac{1}{2}\ln{(1 - x^2)} + C_1$

    $\therefore u = \frac{C_1}{\sqrt{1 - x^2}} \qquad \because u|_{x = 0} = 1 \qquad \therefore C_1 = 1 \qquad$
    $\therefore u = \frac{1}{\sqrt{1 - x^2}}$

    $\because y^{\prime} = u \qquad \therefore \int \mathrm{d}y = \int \frac{1}{\sqrt{1 - x^2}}\mathrm{d}x \qquad$
    $\therefore y = \arcsin{x} + C_2$

    $\because y|_{x = 0} = 0 \qquad \therefore C_2 = 0 \qquad \therefore y = \arcsin{x}$
\end{solution}

\section{选择题:本题共 3 小题,每小题 3 分,共 9 分}

\begin{question}[points = 3]
    设 $(x_0, y_0)$ 是函数 $f(x, y)$ 定义域内的一点,则在点 $(x_0, y_0)$ 处,下列命题正确的是 \paren[C]
\end{question}
\begin{choices}
    \item 若函数 $z = f(x, y)$ 作为任一变量 $x$ 或 $y$ 的一元函数都连续,则 $z = f(x, y)$ 必连续.
    \item 若函数 $z = f(x, y)$ 不连续,则 $z = f(x, y)$ 偏导数必不存在.
    \item 若函数 $z = f(x, y)$ 可微,则 $z = f(x, y)$ 在 $(x_0, y_0)$ 延任一方向的方向导数存在.
    \item 若函数 $z = f(x, y)$ 可微,则必存在一阶连续偏导数.
\end{choices}
\begin{solution}
    考察二元函数性质
    \begin{choices}
        \item 在 $x$ 或 $y$ 方向上连续 $\nRightarrow$ 在任意路径上连续.
        \item 二元函数的连续性与偏导数的存在性无关.
        \item 可微 $\Rightarrow$ 方向导数存在.
        \item 一阶偏导数连续 $\Rightarrow$ 可微,可微 $\nRightarrow$ 一阶偏导数连续.
    \end{choices}
\end{solution}

\begin{question}[points = 3]
    若 $\oiint\limits_\Sigma P\mathrm{d}y\mathrm{d}z + Q\mathrm{d}z\mathrm{d}x + R\mathrm{d}x\mathrm{d}y$
    $= \oiint\limits_\Sigma (P\cos{\alpha} + Q\cos{\beta} + R\cos{\gamma})\mathrm{d}S$ 等式成立,侧 $\cos{\alpha}, \cos{\beta}, \cos{\gamma}$ 是光滑曲面 $\Sigma$ 的 \paren[A]
\end{question}
\begin{choices}
    \item 法向量的方向余弦.
    \item 指向外侧的法向量的方向余弦.
    \item 指向内侧的法向量的方向余弦.
    \item $\Sigma$ 上曲线切向量的方向余弦.
\end{choices}
\begin{solution}
    根据一二两类曲面积分间的联系,令 $\vec{F} = (P, Q, R)$

    $\therefore \oiint\limits_\Sigma \vec{F}\cdot\mathrm{d}\vec{S} = \oiint\limits_\Sigma \vec{F}\cdot\vec{n^0}\mathrm{d}S \qquad$
    $\because$ 滑曲面 $\Sigma$ 的法向量 $\vec{n^0} = (\cos{\alpha}, \cos{\beta}, \cos{\gamma})$

    $\therefore \vec{F}\cdot\vec{n^0} = P\cos{\alpha} + Q\cos{\beta} + R\cos{\gamma} \qquad$
    $\therefore$ 原等式成立;
\end{solution}

\begin{question}[points = 3]
    微分方程 $y^{\prime\prime} - 3y^{\prime} + 2y = xe^{2x}$ 的一个特解应具有形式 \paren[B]
\end{question}
\begin{choices}
    \item $ax^2 + bx$
    \item $(ax^2 + bx)e^{2x}$
    \item $(ax + b)e^{2x}$
    \item $ax + b$
\end{choices}
\begin{solution}
    特征方程 $\lambda^2 - 3\lambda + 2 = 0 \qquad \therefore \lambda_1 = 1 \quad \lambda_2 = 2$

    $\because \mu = 2$ 为特征方程的一个根 $\qquad \therefore Q_m = x(ax + b) = ax^2 + bx \qquad$
    $\therefore y^* = (ax^2 + bx)e^{2x}$
\end{solution}

\section{解答题:本题共 9 小题,共 67 分}
\examsetup
{
    solution/blank-type = manual,
    % 解答空间设置为25±5ex
    solution/blank-vsep = 25ex plus 5ex minus 5ex
}

\begin{problem}[points = 8]
求函数 $f(x, y) = x^3 + 2x^2 + y^2 -2xy$ 的极值.
\end{problem}

\begin{problem}[points = 8]
判断下列级数是收敛还是发散?并说明理由.
\end{problem}
\begin{calculations}[columns = 2, label = (\arabic*)]
    \item $\sum\limits_{n = 1}^{\infty} \frac{2^nn!}{n^n}$
    \item $\sum\limits_{n = 1}^{\infty} \left[\frac{1}{n} - \ln{\left(1 + \frac{1}{n}\right)}\right]$
\end{calculations}

\begin{problem}[points = 8]
判断下列级数是绝对收敛、条件收敛还是发散?并说明理由.
\end{problem}
\begin{calculations}[columns = 2, label = (\arabic*)]
    \item $\sum\limits_{n = 1}^{\infty} (-1)^n\left(1 + \frac{2}{n}\right)^{n^2}$
    \item $\sum\limits_{n = 2}^{\infty} (-1)^{n + 1}\frac{1}{\ln{n}}$
\end{calculations}

\begin{problem}[points = 8]
求 $f(x) = \ln{(2 + x)}$ 的麦克劳林展开式.
\end{problem}

\begin{problem}[points = 8]
计算曲面积分 $\iint\limits_\Sigma (z - x)\mathrm{d}y\mathrm{d}z + (y + y^2)\mathrm{d}z\mathrm{d}x + (z^2 + xy^2)\mathrm{d}x\mathrm{d}y$ ,
其中 $\Sigma$ 为下半球面 $x^2 + y^2 + z^2 = 1(z \leq 0)$ ,取下侧.
\end{problem}

\begin{problem}[points = 8]
求幂级数 $\sum\limits_{n = 0}^{\infty} 4n^2x^{2n - 1}$ 的收敛域与和函数.
\end{problem}

\begin{problem}[points = 5]
证明:设函数 $P(x, y),\, Q(x, y)$ 在单连通区域 $G$ 内具有一阶连续偏导数,若在 $G$ 内,$P\mathrm{d}x + Q\mathrm{d}y$ 是某二元函数 $u(x, y)$ 的全微分,
即 $\mathrm{d}u = P\mathrm{d}x + Q\mathrm{d}y$ ,则在 $G$ 内每一点都有 $\frac{\partial P}{\partial y} = \frac{\partial Q}{\partial x}$.
\end{problem}

\begin{problem}[points = 6]
设 $\Sigma$ 为曲面 $z = x^2 + y^2(0 \leq z \leq 1)$ ,点 $M(x, y, z) \in \Sigma$ ,
$\Pi$ 为 $\Sigma$ 在点 $M$ 处的切平面, $d(x, y, z)$ 为原点到平面 $\Pi$ 的距离,
求 $\iint\limits_\Sigma d(x, y, z)\mathrm{d}S$.
\end{problem}

\begin{problem}[points = 8]
设曲线积分 $\int_L [4f(x) - (2x + 3)\sin{x}]y\mathrm{d}x - f^{\prime}(x)\mathrm{d}y$ 与路径无关,
其中 $f(x)$ 及其导函数 $f^{\prime}(x)$ 过原点,且 $f^{\prime}(x)$ 连续,求 $f(x)$.
\end{problem}
