% @file paper.tex
% @brief 2021高等数学I(2)期末考题
% @author 24bit-xjkp
% @email 2283572185@qq.com

\section{填空题:本题共 10 小题,每小题 3 分,共 30 分}
\scoringbox

\begin{question}[points = 3]
    极限 $\lim\limits_{x \to 0 \atop y \to 1} \frac{1 - xy}{x^2 + y^2} = $ \fillin[1];
\end{question}
\begin{solution}
    分子分母极限均存在,直接计算即可。 $\lim\limits_{x \to 0 \atop y \to 1} \frac{1 - xy}{x^2 + y^2} = 1$;
\end{solution}

\begin{question}[points = 3]
    设函数 $z = e^{xy}$ ,在点 $(1, 1)$ 处当 $\Delta x = 0.15, \Delta y = 0.1$ 时的全增量 $\Delta z = $ \fillin[$0.25e$];
\end{question}
\begin{solution}
    $\because \frac{\partial z}{\partial x} = ye^x \quad \frac{\partial z}{\partial y} = xe^y \qquad
        \therefore dz = ye^xdx + xe^ydy$ \\
    $\therefore dz|_{(1, 1)} = edx + edy \qquad
        \therefore \Delta z = 0.15e + 0.1e = 0.25e$;
\end{solution}

\begin{question}[points = 3]
    曲面 $e^z - z + xy = 3$ 在点 $(2, 1, 0)$ 处的切平面方程为 \fillin[$x + 2y = 4$];
\end{question}
\begin{solution}
    令 $F\left(x, y, z\right) = e^z - z + zy - 3$

    $\therefore \frac{\partial F}{\partial x} = y \quad \frac{\partial F}{\partial y} = x \quad \frac{\partial F}{\partial z} = e^z - 1$

    $\therefore \frac{\partial F}{\partial x}|_{(2, 1, 0)} = 1 \quad$
    $\frac{\partial F}{\partial y}|_{(2, 1, 0)} = 2 \quad$
    $\frac{\partial F}{\partial z}|_{(2, 1, 0)} = 0$

    $\therefore \vec{n} = (1, 2, 0) \qquad$
    $\therefore \Pi: (x - 2) + 2(y - 1) = 0 \qquad$
    $\therefore \Pi: x + 2y = 4$;
\end{solution}

\begin{question}[points = 3]
    设 $\Sigma$ 为球面 $x^2 + y^2 + z^2 = 1$ ,则球面上的点 $\left(\frac{1}{\sqrt{3}}, \frac{1}{\sqrt{3}}, \frac{1}{\sqrt{3}}\right)$
    处指向球面内侧的单位法向量为 \fillin[$\left(\frac{1}{\sqrt{3}}, \frac{1}{\sqrt{3}}, \frac{1}{\sqrt{3}}\right)$];
\end{question}
\begin{solution}
    令 $F\left(x, y, z\right) = x^2 + y^2 + z^2 - 1$

    $\therefore \frac{\partial F}{\partial x} = 2x \quad \frac{\partial F}{\partial y} = 2y \quad \frac{\partial F}{\partial z} = 2z$

    $\therefore \frac{\partial F}{\partial x}|_{\left(\frac{1}{\sqrt{3}}, \frac{1}{\sqrt{3}}, \frac{1}{\sqrt{3}}\right)} = \frac{2}{\sqrt{3}} \quad$
    $\frac{\partial F}{\partial y}|_{\left(\frac{1}{\sqrt{3}}, \frac{1}{\sqrt{3}}, \frac{1}{\sqrt{3}}\right)} = \frac{2}{\sqrt{3}}$
    $\quad \frac{\partial F}{\partial z}|_{\left(\frac{1}{\sqrt{3}}, \frac{1}{\sqrt{3}}, \frac{1}{\sqrt{3}}\right)} = \frac{2}{\sqrt{3}}$

    $\therefore \vec{n} = \left(\frac{2}{\sqrt{3}}, \frac{2}{\sqrt{3}}, \frac{2}{\sqrt{3}}\right) \qquad$
    $\therefore \vec{n^0} = \left(\frac{1}{\sqrt{3}}, \frac{1}{\sqrt{3}}, \frac{1}{\sqrt{3}}\right)$;
\end{solution}

\begin{question}[points = 3]
    设向量场 $A = x(1 + x^2z)\vec{i} + y(1 - x^2z)\vec{j} + z(1 - x^2z)\vec{k}$,
    则其在点 $M(1, 2, -1)$ 处的散度 $\mathrm{div} \, \vec{A}|_M = $ \fillin[3];
\end{question}
\begin{solution}
    $\because \nabla\cdot\vec{M} = (1 + 3x^2z) + (1-x^2z) + (1-2x^2z) = 3$

    $\therefore \mathrm{div} \, \vec{A}|_M = 3$;
\end{solution}

\begin{question}[points = 3]
    若 $|q| < 1$ ,则级数 $\sum\limits_{n = 1}^{\infty} aq^n$ 的和为 \fillin[$\frac{aq}{1 - q}$];
\end{question}
\begin{solution}
    令 $b_n = aq^n \qquad \therefore b_1 = aq \qquad \therefore s_n = \frac{aq(1 - q^n)}{1 - q}$

    $\therefore \sum\limits_{n = 1}^{\infty} aq^n = \lim\limits_{n \to \infty} s_n = \frac{aq}{1 - q}$;
\end{solution}

\begin{question}[points = 3]
    设 $\lim\limits_{n \to \infty} |\frac{a_{n + 1}}{a_n}| = 2$ ,则级数 $\sum\limits_{n = 1}^{\infty} \frac{a_n}{3^n}(x + 3)^n$
    的收敛半径 $R = $ \fillin[$\frac{3}{2}$];
\end{question}
\begin{solution}
    令 $t = x + 3, b_n = \frac{a_n}{3^n} \qquad \therefore s_n = \sum\limits_{n = 1}^{\infty} b_n t^n$

    $\therefore \rho = \lim\limits_{n \to \infty} |\frac{b_{n + 1}}{b_n}| = |\frac{a_{n + 1}}{3 a_n}| = \frac{2}{3} \qquad$
    $\therefore R = \frac{1}{\rho} = \frac{3}{2}$;
\end{solution}

\begin{question}[points = 3]
    $f(x) = \left\{
        \begin{aligned}
             & -x, -\pi < x \leq 0 \\
             & 0, 0 < x \leq \pi
        \end{aligned}
        \right
        .$,
    $s(x)$ 为 $f(x)$ 的以 $2\pi$ 为周期的傅里叶级数的和函数,
    则$s(\pi) + s(-\pi) = $\fillin[$\pi$];
\end{question}
\begin{solution}
    $\because x = \pi$ 是 $f(x)$ 的间断点 $\qquad \therefore s(\pi) = \frac{1}{2}[f(\pi^-) + f(\pi^+)] = \frac{1}{2}[0 + \pi] = \frac{\pi}{2}$

    同理 $s(-\pi) = \frac{\pi}{2} \qquad \therefore s(\pi) + s(-\pi) = \pi$;
\end{solution}

\begin{question}[points = 3]
    微分方程 $y^{\prime\prime} = -\frac{3}{x}y^{\prime} + 1 (x > 0)$ 满足初始条件 $y|_{z = 1} = 0, \, y^{\prime}|_{z = 1} = 1$
    的特解为 \fillin[$y = \frac{x^2}{8} - \frac{3}{8}x^{-2} + \frac{1}{4}$];
\end{question}
\begin{solution}
    令 $f(x) = y^{\prime} \qquad \therefore f^{\prime} = -\frac{3}{x}f + 1 \qquad$
    $\therefore \int\frac{df}{f} = \int -\frac{3}{x} + \frac{1}{f} dx$

    $\therefore \ln f = -3\ln x + \int\frac{1}{f}dx \qquad$
    $\therefore f = x^{-3}C(x)$

    将 $f$ 带回方程 $-3x^{-4}C + x^{-3}C^{\prime} = -3x^{-4}C + 1 \qquad \therefore C^{\prime} = x^3 \qquad$
    $\therefore C(x) = \frac{x^4}{4} + C_1$

    $\therefore f = \frac{x}{4} + C_1x^{-3} \qquad$
    又 $\because f|_{x = 1} = y^{\prime}|_{z = 1} = 1 \qquad$
    $C_1 = \frac{3}{4} \qquad \therefore f = \frac{x}{4} + \frac{3}{4}x^{-3}$

    又 $\because f = y^{\prime} \qquad \therefore \int y dy = \int \frac{x}{4} + \frac{3}{4}x^{-3} dx \qquad$
    $\therefore y = \frac{x^2}{8} - \frac{3}{8}x^{-2} + C_2$

    又 $\because y|_{z = 1} = 0 \qquad \therefore C_2 = \frac{1}{4} \qquad$
    $\therefore y = \frac{x^2}{8} - \frac{3}{8}x^{-2} + \frac{1}{4}$;
\end{solution}

\begin{question}
    已知 $y_1 = xe^x + e^{2x}, \, y_2 = xe^x, \, y_3 = xe^x + e^{2x} - e^{-x}$ 是二阶常系数线性齐次微分方程
    $y^{\prime\prime} + py^{\prime} + qy = e^x - 2xe^x$ 的三个特解,则此微分方程为
    \fillin[$y^{\prime\prime} - y^{\prime} - 2y = e^x - 2xe^x$];
\end{question}
\begin{solution}
    $\because$ 观察特解 $y_1, y_2, y_3$ 可知,对于 $e^{\mu x}$ ,存在 $\mu_1 = 1, \mu_2 = 2, \mu_3 = -1$

    又 $\because$ 非齐次部分 $e^x - 2xe^x = e^x(1 - 2x)$ ,注意到公因式为 $e^x$

    $\therefore$ 特解 $y^*$ 满足 $y^* = e^x(ax + b)$ 形式 $\qquad$
    $\therefore \mu = 1$ 不是特征方程 $\lambda^2 + p\lambda + q = 0$ 的解

    $\therefore$ 特征方程的解 $\lambda_1 = 2, \lambda_2 = -1 \qquad$
    $\therefore$ 特征方程为 $(\lambda - 2)(\lambda + 1) = 0$

    $\therefore \lambda^2 - \lambda - 2 = 0 \qquad$
    $\therefore$ 根据特征方程反推得原微分方程 $y^{\prime\prime} - y^{\prime} - 2y = e^x - 2xe^x$;
\end{solution}

\section{选择题:本题共 3 小题,每小题 3 分,共 9 分}
\scoringbox

\begin{question}[points = 3]
    已知 $f(x, y) = e^{\sqrt{x^4 + y^2}}$ ,则 \,\paren[C]
\end{question}
\begin{choices}
    \item $f_x(0, 0), f_y(0, 0)$ 都存在
    \item $f_x(0, 0), f_y(0, 0)$ 都不存在
    \item $f_x(0, 0)$ 存在, $f_y(0, 0)$ 不存在
    \item $f_x(0, 0)$ 不存在, $f_y(0, 0)$ 存在
\end{choices}
\begin{solution}
    $\because f(0, 0) = e^{\sqrt{0^4 + 0^2}} = 1$

    $\therefore f_x(0, 0) = \lim\limits_{\Delta x \to 0} \frac{e^{\sqrt{{\Delta x}^4 + 0^2}} - f(0, 0)}{\Delta x}$
    $= \lim\limits_{\Delta x \to 0} \frac{e^{{\Delta x}^2} - 1}{\Delta x} = \lim\limits_{\Delta x \to 0} \frac{{\Delta x}^2}{\Delta x} = 0$

    $\therefore f_y(0, 0) = \lim\limits_{\Delta y \to 0} \frac{e^{\sqrt{0^4 + {\Delta y}^2}} - f(0, 0)}{\Delta y}$
    $= \lim\limits_{\Delta y \to 0} \frac{e^{ |\Delta y| } - 1}{\Delta y}$

    又 $\because \lim\limits_{\Delta y \to 0^+} \frac{e^{ |\Delta y| } - 1}{\Delta y} = \lim\limits_{\Delta y \to 0^+} \frac{\Delta y}{\Delta y} = 1 \qquad$
    $\lim\limits_{\Delta y \to 0^-} \frac{e^{ |\Delta y| } - 1}{\Delta y} = \lim\limits_{\Delta y \to 0^-} \frac{-\Delta y}{\Delta y} = -1$

    $\therefore f_x(0, 0)$ 存在, $f_y(0, 0)$ 不存在;
\end{solution}

\begin{question}[points = 3]
    设函数 $f(x, y)$ 为平面上连续函数,则 $\int_{-1}^{1}dx \int_{0}^{1} yf(x^2, y^2)dy = $ \, \paren[A]
\end{question}
\begin{choices}
    \item $2 \int_{0}^{1}dx \int_{0}^{1} yf(x^2, y^2)dy$
    \item $4 \int_{0}^{1}dx \int_{0}^{x} yf(x^2, y^2)dy$
    \item $2 \int_{0}^{1}dy \int_{-y}^{y} yf(x^2, y^2)dy$
    \item $0$
\end{choices}
\begin{solution}
    $\because$ 积分区域 $D$ 为 $x \in [-1 ,1], y \in [0, 1]$ ,关于 $y$ 轴对称

    又 $\because$ 被积函数 $yf(x^2, y^2)$ 是 $x$ 的奇函数

    $\therefore \int_{-1}^{1}dx \int_{0}^{1} yf(x^2, y^2)dy = 2 \int_{0}^{1}dx \int_{0}^{1} yf(x^2, y^2)dy$;
\end{solution}

\begin{question}[points = 3]
    设 $\mu_n > 0 (n = 1, 2, \cdots)$ ,若 $\lim\limits_{n \to \infty} n^2 \mu_n = l (0 < l < + \infty)$ ,
    则交错级数 $\sum\limits_{n = 1}^{\infty} (-1)^n \mu_n$ \, \paren[A]
\end{question}
\begin{choices}
    \item 绝对收敛
    \item 条件收敛
    \item 发散
    \item 不能确定其敛散性
\end{choices}
\begin{solution}
    $\because \lim\limits_{n \to \infty} n^2 \mu_n = \lim\limits_{n \to \infty} \frac{\mu_n}{\frac{1}{n^2}} = l (0 < l < + \infty)$

    $\therefore \sum\limits_{n = 1}^{\infty} \mu_n$ 的敛散性与 $\sum\limits_{n = 1}^{\infty} \frac{1}{n^2}$ 相同 $\qquad$
    又 $\because \sum\limits_{n = 1}^{\infty} \frac{1}{n^2}$ 收敛 $\qquad$
    $\therefore \sum\limits_{n = 1}^{\infty} \mu_n$ 收敛

    $\therefore \sum\limits_{n = 1}^{\infty} (-1)^n \mu_n$ 绝对收敛
\end{solution}
