% @file 高等数学2021-I(2)-期末.tex
% @brief 2021高等数学I(2)期末考题
% @author 24bit-xjkp
% @email 2283572185@qq.com

\documentclass{exam-zh}

\examsetup{
    title = {title-format= \bfseries\huge},
    fillin/type = line,
    fillin/no-answer-type = none,
    question/show-answer = false,
    solution/show-solution = show-move
}

\title{南京航空航天大学}
\subject{2021《高等数学I(2)》期末考试试题}

\begin{document}
\maketitle

\section{填空题(每空3分),共30分}
\scoringbox

\begin{question}[points = 3]
    极限 $\lim\limits_{x \to 0 \atop y \to 1} \frac{1 - xy}{x^2 + y^2} = $ \fillin[1];
\end{question}
\begin{solution}
    分子分母极限均存在,直接计算即可。$\lim\limits_{x \to 0 \atop y \to 1} \frac{1 - xy}{x^2 + y^2} = 1$;
\end{solution}

\begin{question}[points = 3]
    设函数$z = e^{xy}$,在点$\left(1, 1\right)$处当$\Delta x = 0.15, \Delta y = 0.1$时的全增量$\Delta z = $ \fillin[$0.25e$];
\end{question}
\begin{solution}
    $\because \frac{\partial z}{\partial x} = ye^x \quad \frac{\partial z}{\partial y} = xe^y \qquad
        \therefore dz = ye^xdx + xe^ydy$ \\
    $\therefore dz|_{\left(1, 1\right)} = edx + edy \qquad
        \therefore \Delta z = 0.15e + 0.1e = 0.25e$;
\end{solution}

\begin{question}[points = 3]
    曲面$e^z - z + xy = 3$在点$\left(2, 1, 0\right)$处的切平面方程为 \fillin[$x + 2y = 4$];
\end{question}
\begin{solution}
    令$F\left(x, y, z\right) = e^z - z + zy - 3$ \\
    $\therefore \frac{\partial F}{\partial x} = y \quad \frac{\partial F}{\partial y} = x \quad \frac{\partial F}{\partial z} = e^z - 1$ \\
    $\therefore \frac{\partial F}{\partial x}|_{\left(2, 1, 0\right)} = 1 \quad
        \frac{\partial F}{\partial y}|_{\left(2, 1, 0\right)} = 2 \quad
        \frac{\partial F}{\partial z}|_{\left(2, 1, 0\right)} = 0$ \\
    $\therefore \vec{n} = \left(1, 2, 0\right) \qquad
     \therefore \Pi: \left(x - 2\right) + 2\left(y - 1\right) = 0 \qquad
     \therefore \Pi: x + 2y = 4$;
\end{solution}

\end{document}
